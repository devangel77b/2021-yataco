\documentclass[10pt]{wrceletter}


\name{Dennis Evangelista}
\position{Former Assistant Professor}
%\email{\href{mailto:evangeli@usna.edu}{\emph{evangeli@usna.edu}}}
\email{devangel@alum.mit.edu}
\telephone{862-360-4201}

\date{\today}

\usepackage{designature}
\signature{\vspace*{-0.5in}\includesignature\\Dennis Evangelista}
%\signature{Dennis Evangelista\\Assistant Professor} % title not needed if in letterhead
\address{\null} %{105 Maryland Avenue\\Annapolis, MD 21402} % leave blank, provided in letterhead
%\longindentation=0in % to change signature to be flushleft

\begin{document}
\begin{letter}{% recipient address here
}

% opening here
\opening{To the Post-Baccalaureate Prograrm (PREP) admissions committee at the University of Pennsylvania:}
\raggedright % if you like this sort of thing
\setlength{\parindent}{15pt} % if you like this sort of thing

I am happy to recommend Jocelyne Yataco for the Post-Baccalaureate Program (PREP) at the University of Pennsylvania. Jocelyne was a student in my department and was in two classes I taught in, EW200 (Introduction to Systems Engineering) and EW202 (Principles of Mechatronics). These two classes are taught to all Robotics and Controls Engineering majors during their 3/c (sophomore) year at the Naval Academy and cover programming, engineering design, and embedded microcontrollers.  

I remember vividly interacting with Jocelyne when she came on several occasions for extra instruction. In EW200, she worked hard to learn objected-oriented programming principles applied to graphical user interfaces (GUI) in Matlab, even though the rudimentary ``video game'' project she was assigned only required very basic plotting commands. I also distinctly remember her keeping me on my toes at exam reviews for EW200 and EW202, where she would attend and ask very insightful questions about C programming, basic circuits, embedded microcontrollers, servos, and motors; showing all the work ethic of a top engineering student. Jocelyne was slated to be in my section for EW485 (Comparative Biomechanics) when she was separated from the Naval Academy due to medical issues which ultimately resulted in her honorable discharge. Jocelyne is about to complete her undergraduate degree in electrical engineering and has continued to chase opportunities such as at JPL where she worked on remote sensing of extreme weather conditions from the stratosphere. The skills involved in pursuing high technology, high risk applications like flight could easily apply to medical device design, neurological implants, and other technologies.

Many people would just quit in face of a life-altering diagnosis.  Instead, Jocelyne has found added strength and motivation to add pre-med studies to her work. With her electrical engineering background, I believe she could be a strong contributor in interdisciplinary work touching on signals, systems, control, neurophysiology, psychiatry, and mental health. As an electrical engineer who pursued a biology/comparative biomechanics PhD at UC Berkeley myself, I have worked with undergrads who combined engineering and medicine and have seen them thrive, including at UPenn. Jocelyne's technical training is excellent preparation for understanding nerve and muscle action potential; the complex networks of inhibition and excitation between neurons; bases of neural computation; and the physics one would use to sense, diagnose, and reset or restore function. Naval Academy mids never give up the ship; Jocelyne's resilience bodes well for her future as a medical student. 

As a US Navy veteran who has experienced a service-connected health condition myself, I have also seen colleagues enter medicine from more circuitous routes to the MD degree, and I believe it to be incredibly beneficial for the field and for the patients. Physicians who can empathize with patients from a deep, personal place make an incredible difference to outcomes. Jocelyne has my highest recommendation for your program. I have seen your program is outstanding for providing clinical and research experience to such candidates, helping medical schools see their strengths coming into MD programs. 

\closing{Very respectfully,} % provides empty 

%\ps{post script here}
%\encl{enclosure here}
\end{letter}
\end{document}


% I would maybe suggest cutting back on the first paragraph a little bit more, and elaborating more on her resilience, work ethic, and interest in medicine - all of which are strong indicators for a successful applicant.
